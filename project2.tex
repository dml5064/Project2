% Options for packages loaded elsewhere
\PassOptionsToPackage{unicode}{hyperref}
\PassOptionsToPackage{hyphens}{url}
\documentclass[
]{article}
\usepackage{xcolor}
\usepackage[margin=1in]{geometry}
\usepackage{amsmath,amssymb}
\setcounter{secnumdepth}{-\maxdimen} % remove section numbering
\usepackage{iftex}
\ifPDFTeX
  \usepackage[T1]{fontenc}
  \usepackage[utf8]{inputenc}
  \usepackage{textcomp} % provide euro and other symbols
\else % if luatex or xetex
  \usepackage{unicode-math} % this also loads fontspec
  \defaultfontfeatures{Scale=MatchLowercase}
  \defaultfontfeatures[\rmfamily]{Ligatures=TeX,Scale=1}
\fi
\usepackage{lmodern}
\ifPDFTeX\else
  % xetex/luatex font selection
\fi
% Use upquote if available, for straight quotes in verbatim environments
\IfFileExists{upquote.sty}{\usepackage{upquote}}{}
\IfFileExists{microtype.sty}{% use microtype if available
  \usepackage[]{microtype}
  \UseMicrotypeSet[protrusion]{basicmath} % disable protrusion for tt fonts
}{}
\makeatletter
\@ifundefined{KOMAClassName}{% if non-KOMA class
  \IfFileExists{parskip.sty}{%
    \usepackage{parskip}
  }{% else
    \setlength{\parindent}{0pt}
    \setlength{\parskip}{6pt plus 2pt minus 1pt}}
}{% if KOMA class
  \KOMAoptions{parskip=half}}
\makeatother
\usepackage{graphicx}
\makeatletter
\newsavebox\pandoc@box
\newcommand*\pandocbounded[1]{% scales image to fit in text height/width
  \sbox\pandoc@box{#1}%
  \Gscale@div\@tempa{\textheight}{\dimexpr\ht\pandoc@box+\dp\pandoc@box\relax}%
  \Gscale@div\@tempb{\linewidth}{\wd\pandoc@box}%
  \ifdim\@tempb\p@<\@tempa\p@\let\@tempa\@tempb\fi% select the smaller of both
  \ifdim\@tempa\p@<\p@\scalebox{\@tempa}{\usebox\pandoc@box}%
  \else\usebox{\pandoc@box}%
  \fi%
}
% Set default figure placement to htbp
\def\fps@figure{htbp}
\makeatother
\setlength{\emergencystretch}{3em} % prevent overfull lines
\providecommand{\tightlist}{%
  \setlength{\itemsep}{0pt}\setlength{\parskip}{0pt}}
\usepackage{bookmark}
\IfFileExists{xurl.sty}{\usepackage{xurl}}{} % add URL line breaks if available
\urlstyle{same}
\hypersetup{
  pdftitle={Project 2 -- Signature creation and iLINCS API},
  hidelinks,
  pdfcreator={LaTeX via pandoc}}

\title{Project 2 -- Signature creation and iLINCS API}
\author{}
\date{\vspace{-2.5em}}

\begin{document}
\maketitle

This assignment builds upon the R/shiny class and expands the API
example.

\begin{enumerate}
\def\labelenumi{\arabic{enumi}.}
\tightlist
\item
  For the assignment use the dataset
  TCGA\_breast\_cancer\_ERstatus\_allGenes.tsv. You may choose to
  ``debug'' your code with a smaller dataset
  TCGA\_breast\_cancer\_ERpositive\_vs\_ERnegative\_PAM50.tsv or
  TCGA\_breast\_cancer\_LumA\_vs\_Basal\_PAM50.tsv.
\item
  Your assignment is to develop code using R Shiny to present user
  interface that allows a user to upload a tsv file, presents an option
  to split the samples into two groups based on the available metadata
  and create a signature. The signature is then submitted to the iLincs
  API which retrieves concordant signatures.
\item
  Template is provided in the Project 2 github project
  (\url{https://github.uc.edu/uc-datascience/Project2.git}).
\item
  The template is missing the calculation of the differential expression
  -- please use t-test to calculate t-statistic / p-value.
\item
  Expand the template to allow users to filter the input file to L1000
  genes only (See the include L1000.txt file).
\item
  Further expand the template to allow users to submit only top 100
  differentially expressed genes.
\item
  Compare results with iLincs
\item
  Extra credit for a heatmap or other visualizations.
\end{enumerate}

The assignment is due on -- March 6, 2026 midnight.

The submission should be zip compressed file named
``project2-{[}\emph{your UC username}{]}.zip''
(e.g.~``project2-lastnfi.zip'') which includes any supporting R files.
The zip file should be uploaded canopy. The assignment entry in Canopy
will be created shortly.

\end{document}
